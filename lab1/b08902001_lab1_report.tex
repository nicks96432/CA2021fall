\documentclass{article}

\usepackage[utf8]{inputenc}
\usepackage[
    a4paper, top=2.5cm, bottom=2.5cm, left=2.5cm, right=2.5cm
]{geometry}
\usepackage{minted}
\usepackage{xeCJK}

\setCJKmainfont{Noto Sans CJK TC}
\setmainfont{Noto Sans CJK TC}

\setlength{\parindent}{0cm}
\renewcommand{\baselinestretch}{1.5}

\begin{document}

\begin{enumerate}
    \item[1.] Control \& ALU\textunderscore Control

        相比Homework 4,這次的Control多了從Hazard Detection Unit來的
        \mintinline{text}{NoOp_i},還有接到Data Memory相關的幾個control
        signal。這次的Control Unit在收到opcode全部都是0 (出現在pipeline flush)
        或是\mintinline{text}{NoOp_i}的時候把所有輸出的訊號都設為0,否則就根據
        opcode的內容設定相應的control signal。

        ALU Control則和上次大同小異,同樣是根據\mintinline{text}{ALUOp_i}和
        \mintinline{text}{funct_i}來決定\mintinline{text}{ALUCtrl_o},不過
        這次多了判斷load和store instruction的\mintinline{text}{funct3}。

    \item[2.] Pipeline Registers

        我把四組Pipeline Registers放在\mintinline{text}{pipeline_registers}
        資料夾下,並從\mintinline{text}{CPU.v} include這四個module。這4組
        registers會在程式開始時的reset訊號產生時將各自的register重設成0。每次到
        clock的positive edge就把input用unblocking的方式存到對應的register。

        其中的IF\textunderscore ID\textunderscore REGISTER比較特別,它還會從
        Hazard Detection Unit讀要不要Stall的訊號,以及從判斷要不要branch的AND
        Gate讀Flush訊號。如果要Stall就不會把register的內容換成新的,如果要Flush
        就會把register歸零。

    \item[3.] Forwarding Unit

        它會讀EX\textunderscore MEM\textunderscore REGISTER的rd和
        ID\textunderscore EX\textunderscore REGISTER的rs1或rs2是否相同,
        如果相同就把EX\textunderscore MEM\textunderscore REGISTER的
        \mintinline{text}{ALUResult} forward給相對應的ALU輸入位置,這樣就可以
        避免EX hazard。

        如果沒有EX hazard,那就用類似的方法判斷有沒有MEM hazard,並在正確的時機
        把WB階段的\mintinline{text}{WriteData} forward給ALU。

    \item[4.] Hazard Detection Unit

        在Hazard\textunderscore Detection.v。它會讀EX階段的instruction
        是否要MemRead和rd,以及ID階段的rs1和rs2,如果EX需要讀記憶體,而且rd
        和ID階段的rs1或rs2相同時,就輸出stall訊號給IF\textunderscore ID
        \textunderscore REGISTER,來避免load-use hazard。

    \item[5.] testbench

        我沒有修改testbench,因為我把pipeline register initialization放在
        它們各自的module裡面。

    \item[6.] Others

        \begin{enumerate}
            \item[(a)] Imm\textunderscore Gen

                負責從instruction把immediate值讀出來進行sign extension並傳給ALU
                還有負責branch instruction的Adder。

            \item[(b)] defs

                這個檔案裡定義了各種opcode,以及ALU control的signal。

                \pagebreak

            \item[(c)] Misc.

                \begin{enumerate}
                    \item[i.] MUX32\textunderscore 2

                        和Homework 4的MUX一樣,會根據輸入的訊號決定輸出
                        \mintinline{text}{data1}或\mintinline{text}{data2}的其中一個

                    \item[ii.] MUX32\textunderscore 4

                        這個MUX會根據兩個bit的訊號,從4個輸入中決定要輸出的一個。使用3個
                        MUX32\textunderscore 2連接而成。

                    \item[iii.] Add32\textunderscore 2

                        這個circuit會讀入兩個數字,並輸出它們的和,不考慮溢位。

                    \item[iv.] Equal32\textunderscore 2

                        這個circuit會讀入兩個數字,並輸出兩數是否相等。

                    \item[v.] SL1\textunderscore 32

                        這個circuit會讀入一個數字,並輸出其bit shift left 1的值。

                \end{enumerate}
        \end{enumerate}

    \item[7.] CPU

        在這裡把所有東西根據spec的datapath接在一起,並且把reset接到各個pipeline
        register上。判斷是否要flush的AND Gate也在這裡,因為verilog的AND gate
        直接用assign語法就可以做出來了,所以我沒有放在額外的檔案裡。在testbench
        的Flush也是接到這個AND Gate上。
\end{enumerate}

\end{document}
