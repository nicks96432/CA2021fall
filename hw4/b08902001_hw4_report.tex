\documentclass{article}

\usepackage[utf8]{inputenc}
\usepackage{xeCJK}
\usepackage[
    a4paper,top=2.5cm,bottom=2.5cm,left=2.5cm,right=2.5cm
]{geometry}
\usepackage{minted}

\setCJKmainfont{Noto Sans CJK TC}
\setmainfont{Noto Sans CJK TC}

\setlength{\parindent}{0cm}
\renewcommand{\baselinestretch}{1.5}

\begin{document}

\section*{Modules Explanation}

\begin{enumerate}
    \item Adder

          這個circuit會讀兩個32 bit的整數,然後輸出兩者的和,不考慮溢位。

    \item ALU\textunderscore Control

          這個circuit會讀一個10 bit的\mintinline{text}{funct}和2 bit的
          \mintinline{text}{ALUOp},\mintinline{text}{funct}由instruction
          的\mintinline{text}{funct7}和\mintinline{text}{funct3}接在一起而成,
          \mintinline{text}{ALUOp}則是Control的輸出之一。ALU\textunderscore
          Control會先根據\mintinline{text}{ALUOp}是10還是00判斷instruction是
          R-Type還是I-Type,然後再根據\mintinline{text}{funct}決定要輸出什麼
          3 bit \mintinline{text}{AluCtrl}給ALU。

    \item ALU

          這個circuit會讀入兩個32 bit的有號整數,及ALU\textunderscore Control
          輸出的3bit \mintinline{text}{AluCtrl},然後根據
          \mintinline{text}{AluCtrl}決定要輸出以輸入的兩個整數的哪種運算結果,
          不考慮溢位。有ADD、SUB、MUL、AND、XOR、SLL、SRA這7種運算。另外還會輸出Zero,
          因為在這個作業不會用到,所以我就把它設為0了。在實作\mintinline{asm}{srai}的
          時候,因為immediate只有5個bit,所以要把第二個整數input和11111做bit and
          ,不然會錯。

    \item Control

          這個circuit會從instruction中讀7 bit的\mintinline{text}{opcode},然後
          根據\mintinline{text}{opcode}代表的instruction來輸出要給ALU\textunderscore
          Control、ALU、register的訊號。如果\mintinline{text}{opcode}代表的是
          R-Type instruction,就把\mintinline{text}{ALUOp}設為10並把
          \mintinline{text}{ALUSrc}恆設為0。如果是I-Type instruction,則把
          \mintinline{text}{ALUOp}設為00並把\mintinline{text}{ALUSrc}設為1。
          因為這次的作業只有這兩個type的instruction,而且都需要寫入register,所以我把
          \mintinline{text}{RegWrite}恆設為1。
    \item CPU

          在這裡根據spec的datapath (Figure 1)把所有的module接在一起。根據每條在datapath
          上的線(有分叉的算同一條)都宣告一條wire,然後在\mintinline{text}{CPU.v}裡面的
          括號填入對應的wire。我在這裡還加了一個wire four恆等於4,專門送到Adder裡面幫PC
          做加法。

    \item MUX32

          這個circuit的input有2個32 bit的整數和1 bit的\mintinline{text}{select}。
          如果\mintinline{text}{select}是0的話就輸出第一個整數,否則輸出第二個整數。

    \item Sign\textunderscore Extend

          這個circuit會從instruction的最左邊12 bit讀入整數,然後根據它的正負號決定
          要在most significant bit的地方補20個0或1,輸入是正數就補0,負就補1。
\end{enumerate}

\section*{Develop Environment}

OS: WSL 2 (Arch Linux 5.10.60.1-microsoft-standard-WSL2) @ windows 11 21H2 (build 22000.318)

Compiler: iverilog 11.0 (stable)

\end{document}